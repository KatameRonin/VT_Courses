\documentclass[11pt]{article}
\usepackage[utf8]{inputenc}
\usepackage{hyperref}
\usepackage{biblatex}

\title{\textbf{Publishing Venues and Peer Reviews}}
\author{Amartya Dutta}
\date{}


\usepackage[letterpaper, margin=1in]{geometry}
\addbibresource{ref.bib}

\begin{document}
\clearpage\maketitle
\thispagestyle{empty}
\section{Publishing Venues}
My research area is Computer Vision and Deep Learning. It has several publication venues divided into tier levels. While the division of the venues into these tiers is subjective and varies from one research group to another. A list of some venues from each tier level is mentioned as follows: \newline
\item{\textbf{Tier 1:}} \newline
\href{https://cvpr.thecvf.com/}{Conference on Computer Vision and Pattern Recognition (CVPR)} \newline
\href{https://iccv2023.thecvf.com/}{International Conference on Computer Vision (ICCV)} \newline
\href{https://eccv2022.ecva.net/}{European Conference on Computer Vision (ECCV)} \newline
\item{\textbf{Tier 2:}} \newline
\href{https://www.sciencedirect.com/journal/medical-image-analysis}{Medical Image Analysis} \newline
\href{https://wacv2023.thecvf.com/home}{Winter Conference on Applications of Computer Vision (WACV)} \newline
\href{https://britishmachinevisionassociation.github.io/bmvc}{British Machine Vision Conference (BMVC)} \newline
\item{\textbf{Tier 3:}} \newline
\href{https://2023.ieeeicip.org/#:~:text=IEEE%20ICIP%20is%20the%20world's,video%20processing%20and%20computer%20vision.}{International Conference on Image Processing (ICIP)} \newline
\href{https://www.accv2022.org/en/}{Asian Conference on Computer Vision (ACCV)} \newline
\href{https://www.icpr2022.com/}{International Conference on Pattern Recognition (ICPR)} \newline
\item{\textbf{Tier 4:}} \newline
\href{https://bioimaging.scitevents.org/}{BioImaging} \newline
\href{https://visapp.scitevents.org/}{International Conference on Computer Vision Theory and Applications} \newline
\href{https://www.springer.com/journal/11042}{Multimedia Tools and Applications} \newline

\section{Research Goals}
I am currently pursuing my MS in Computer Science. This implies that my Master's has a thesis focus. Accordingly, I spoke to my guide, and although this may not be true for all, we wish to aim big and publish at least 1 paper in a Tier 1 conference and try for around 2 in Tier 2 conferences through the span of my Master's program. While my Master's Committee does urge me to have publications, these are not strictly the requirements for them to grant me my degree. However, these are the goals that I share with my professor. 

\section{Paper Review}
A research paper that is very relevant to my field of research is titled "Anti-Adversarially Manipulated Attributions for Weakly and Semi-Supervised Semantic Segmentation". This paper was published in CVPR 2021, a Tier 1 conference for Computer Vision. \newline
In the paper \cite{lee2021anti}, the authors present a method for improving the interpretability and robustness of weakly and semi-supervised semantic segmentation models. The proposed method involves training a second model to generate anti-adversarial perturbations that can be applied to the input of the original model. These perturbations are designed to prevent the original model from relying on spurious correlations or shortcuts in the input data, which can lead to incorrect predictions or vulnerabilities to adversarial attacks. By analyzing the attributions generated by the perturbed input data, the authors demonstrate that their method improves the interpretability of the original model and can help identify areas of the input that are important for accurate segmentation. \newline
Overall, the paper is well-written and presents a novel method that addresses an important problem in weakly and semi-supervised semantic segmentation. Its strength lies in the novelty and significance of the proposed method. The authors address an important problem in weakly and semi-supervised semantic segmentation, namely the reliance of models on spurious correlations in the input data. The method they propose involves the use of anti-adversarial perturbations, which is a creative and effective way to prevent this problem and improve model interpretability. However, there are some areas where the organization of the paper could be improved. For example, the introduction could be clearer about the motivation and significance of the proposed method, and the related work section could be better integrated with the rest of the paper.  The paper is also correct and makes claims that are sufficiently justified by data and references. The authors also provide a thorough evaluation of their method, including comparisons to several baseline approaches and analysis of the attributions generated by their method. A weakness however is that the evaluation is mostly focused on quantitative metrics and does not include a detailed analysis of specific examples or qualitative feedback from users. It would be useful to see how the proposed method performs on challenging real-world examples, such as those involving occlusions, ambiguous boundaries, or complex scenes. A crucial piece of information that the paper is missing out on is that the authors do not provide enough detail on the architecture and hyperparameters of their models, making it difficult to reproduce their results or extend their method to other tasks. It would be helpful to have a more comprehensive description of the training and evaluation protocols used in the experiments. Another issue is that the authors do not discuss the computational cost of their method, which may be a concern for large-scale or real-time applications. Overall,  while there are some minor issues that should be addressed, the paper makes a valuable and novel contribution to the literature which improves the baseline score in the field of Weakly Supervised Semantic Segmentation. Thus, the reviewers did justice to this work by accepting it to a Tier 1 conference. \newline

\printbibliography
\end{document}