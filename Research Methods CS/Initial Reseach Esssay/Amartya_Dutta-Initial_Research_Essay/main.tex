\documentclass[11pt]{article}
\usepackage[utf8]{inputenc}

\title{\textbf{Initial Research Essay}}
\author{Amartya Dutta}
\date{}


\usepackage[letterpaper, margin=1in]{geometry}

\begin{document}

\clearpage\maketitle
\thispagestyle{empty}

\section{Introduction and Background}
I am Amartya Dutta and I completed my BTech in Computer Science and Engineering from the Indian Institute of Information Technology (IIIT) in Guwahati, India. IIIT Guwahati is a Central Government College where diverse people from all over India come together to study. This gave me great exposure to people from different walks of life and taught me to work with diverse people. This experience was necessary because I would certainly have to work with people from different cultural backgrounds, so learning early on would be helpful for my future. \par
During my undergraduate studies, I had to complete courses that spanned various areas of  Computer Science. These included subjects like Algorithms, Computer Architecture, Automata Theory, Operating Systems, Networks, Security, Cloud Computing, Artificial Intelligence, and even Human-Computer Interaction. Besides these, there were also important courses in Mathematics such as Calculus, Linear Algebra, and Discrete Mathematics. Overall, my undergraduate courses gave me an idea of the different possible fields in Computer Science, making it easier to find the fields that interested me. \par
In my Sophomore year, I developed an interest in Human-Computer Interaction, specifically in Augmented and Virtual Reality. Accordingly, I worked on personal projects that involved Virtual and Augmented Reality tours. I was specifically interested in developing interactive environments for the users. In order to further my understanding of the field, I decided to intern as a researcher with an aim to improve user experience in Virtual Reality, at the User-Centric Computing and Networking Lab at the Indian Institute of Technology Guwahati. During my internship, I worked with Graduate students and also read research papers for the first time. We performed several experiments time and over; it was a great learning experience. This encounter developed a deep appreciation for research in me. I tried to gain further experience by interning as an Augmented Reality Developer at a startup. We developed products for the users and to understand their desires better we also conducted surveys. Around the same time, I also took part in several National Level Hackathons and competed against thousands of teams to better understand how people perceived my project ideas. All these got me excited about learning and understanding what people expect from an application they use in order to have a good experience. However, working on these topics over time led me to find out that all my developed applications lacked automation in terms of detection and other features. So, given the exposure, I had from my courses I started following researchers working on Artificial Intelligence. This led me to deep dive into Artificial Intelligence because I learned that it would grant my work a lot more flexibility, allowing me to work across different fields, thus improving a user’s experience or even making their lives better. \par
I undertook AI-based research work under one of my University Professors. Eventually, after working extensively during the pandemic period, I submitted my work to an international conference. I learned a lot about the field by working and also listening to others who presented their work at the conference. I also wanted to work with other people in different areas, so I collaborated with the Civil Engineering Department of IIT Guwahati, where we used Deep Learning methods to monitor the health of a bridge efficiently. This experience opened my eyes to new opportunities altogether. Following this, I undertook a project that was a part of another international conference where we used Deep Learning models to detect medical issues related to the gastrointestinal tract faster and better. Though I learned how complex medical data could be, at the same time it also instilled in me hope that I could impact the lives of people. I had chosen to work on medical images because, on several occasions, I have felt helpless when relatives succumbed to illnesses and therefore, I wanted to give myself hope that I too could contribute something to the medical field. \par
I've had an ardent interest in Research since my undergraduate years, therefore to further explore my field of interest and learn to work on challenging problems, I considered pursuing a Graduate degree. I chose to do a Master's degree because a Ph.D. is more of a specialization and I believe I'm yet to find a topic in which I am completely invested. Eventually, I shortlisted Virginia Tech. as one of my choice of schools because of its reputation as a Top Tier Research University and the exciting research on Artificial Intelligence in some of the labs.  


\section{Research Progress}

I am currently pursuing my Master's (MS) in Computer Science at Virginia Tech.  Given the requirements of the MS program, I have to complete a total of seven courses and two graduate seminars of which I have completed two courses and one seminar in my first semester and have taken two of the remaining five courses in my current semester. \par
The next step for me was to find an advisor and join a lab early on so that I could also start working on my Thesis on time. Therefore, I attended a seminar where all the Faculties talked about their research interests and if they had openings in the lab. Accordingly, I shortlisted the Faculties that could fund me in the future and whose research areas matched what I had in mind. I emailed professors expressing my interest in joining their labs and how my past experiences would be relevant for conducting successful research work. Eventually, I joined the Knowledge Guided Machine Learning (KGML) Lab under the guidance of Dr. Anuj Karpatne. I chose this lab because the main philosophy behind it is that even though Machine Learning has become a very popular field as of late and there have been many popular architectures that perform well when provided with a lot of data, sometimes the performance of these models cannot be explained or they do not perform well, in terms of the fact that they do not satisfy the existing constraints or rules that govern the data. This is because these models often termed “black box models” rely solely on data and are not rooted in scientific knowledge. Hence, this resulted in a growing interest in the scientific community in creating a new generation of methods that integrate scientific knowledge in ML frameworks. This emerging field, called scientific Knowledge-guided Machine Learning (KGML), seeks a distinct departure from existing “data-only” or “scientific knowledge-only” methods to use knowledge and data on an equal footing. \par
I have been a part of the Lab since September 2022 and since then I've been focusing on a project that deals with Weakly Supervised Semantic Segmentation of Objects. The idea is that annotated data is costly and scarce so we want to train a Machine learning model to successfully segment out objects even when no manual annotation is present. Therefore, we have been looking into the current State-Of-The-Art works in this field and accordingly carrying out experiments on the benchmark datasets with our own ideas in order to beat the target scores. I have been working on this along with several other Ph.D. students in the lab and it has been a great learning experience yet. Besides this, I am also a part of another Research Project titled Imageomics, which is collaborative work and involves people from several universities and industries.  Imageomics is a new idea in biology where images are used as the source of information about life and are powered by novel advances in knowledge-guided ML (KGML). We are developing methods that make use of varying forms of structured biological knowledge to guide the training of ML models on images of organisms for a variety of tasks such as species classification, image reconstruction, and trait discovery.

\section{Aspirations and Future Plans}

I completed my Bachelor's in 2021 and joined  Virginia Tech. for my Master's in 2022. For a year since graduation, I was involved in academic research. I have not had any experience working in the industry, so my next plan after completing my Master's is to get involved in Industry. I want to join as a Machine Learning engineer focusing specifically on Computer Vision problems. While my graduate experience will prepare me with the knowledge to work on Machine Learning problems, I want to grow as a Developer as well, because in the industry simply writing code doesn't do. The code needs to be optimized and then deployed so that the customers can avail of the services. Therefore, I believe a lot of engineering needs to be done, and learning these skills is important besides just being able to conduct research. \par
Eventually, my aim is to be a Research Scientist in Industry. But I want to be a successful Research and Development (R\&D) Scientist and two companies that are my goal are Google and Facebook. Specifically, I want to be a part of Facebook AI Research (FAIR) and Google's DeepMind. The research performed at both these labs is cutting-edge and I want to be involved in working on such problems. However, having just a Master's degree and Industry Experience is not sufficient to reach what I aim for. One needs to have a good publication record in top-tier conferences or journals to be eligible for being a Research Scientist in such companies. However, the time duration of a Master's is not sufficient to generate such successful publications. This is the reason, after a few years of Industry experience I also plan to enroll in a Ph.D. program. While I am aware that pursuing Ph.D. right after my Master's is more time efficient but I want to be well prepared in terms of both Research and Development and that is why I feel an Industry experience will be crucial. In my Ph.D., I plan to work on a problem statement I encounter while working on my Master's Thesis. Finally, while the mental road map I have regarding my future is bound to be shaped better by my future experiences, I believe I shall follow the path I must to achieve my goals. 
\end{document}
